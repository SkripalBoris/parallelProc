\documentclass[12pt,a4paper]{report}

\usepackage[T2A]{fontenc}
\usepackage[utf8]{inputenc}
\usepackage[english,russian]{babel}
\usepackage{amsmath}
\usepackage{amsfonts}
\usepackage{amssymb}
\usepackage{graphicx}
\usepackage{listings}

\voffset -24.5mm
\hoffset -5mm
\textwidth 173mm
\textheight 240mm
\oddsidemargin=0mm \evensidemargin=0mm

\lstdefinestyle{CppCodeStyle}{
	basicstyle=\footnotesize\ttfamily,
	language={[ANSI]C++},
	keywordstyle=\bfseries,
	showstringspaces=false,
	morekeywords={include, printf},
	commentstyle={},
	texcl=true,
	frame=single,
	breaklines=true,
	extendedchars=\true
}

\begin{document}
	\begin{titlepage}
\begin{center}

\textbf{Санкт-Петербургский политехнический университет Петра Великого}

\vspace{5mm}
Институт компьютерных наук и технологий

\vspace{5mm}
Кафедра компьютерных систем и программных технологий

\vspace*{\fill}

\huge{ОТЧЕТ}

\Large{о лабораторной работе №1}

\large{по дисциплине: <<Параллельные вычисления>>}

\vspace*{2mm}
\large{Тема работы: <<Подсчет количества русских слов в тексте>>}

\vspace*{\fill}
\end{center}

%\begin{flushright}
\begin{large}
\hspace{0.4\linewidth} \textbf{Работу выполнил студент}

\vspace{5mm}
\hspace{0.4\linewidth} 53051/3 \hspace{1cm} \textit{Скрипаль Б.А.}

\vspace{3mm}
\hspace{0.4\linewidth} \textbf{Преподаватель}

\vspace{5mm}
\hspace{0.4\linewidth} \underline{\hspace{2cm} } \hspace{3mm} \textit{Стручков И.В.}
\end{large}
%\end{flushright}

\vspace*{3cm}

\begin{center}
\normalsize Санкт-Петербург\\2016
\end{center}
\end{titlepage}
	
	\renewcommand{\thesection}{\arabic{section}}
	\tableofcontents
	\pagebreak
	
	\setcounter{totalnumber}{10}
	\setcounter{topnumber}{10}
	\setcounter{bottomnumber}{10}
	\renewcommand{\topfraction}{1}
	\renewcommand{\textfraction}{0}
	
	\section{Постановка задачи}
		Реализовать программу, подсчитывающую количество русских слов в тексте при помощи трех технологий:
		\begin{enumerate}
			\item Последовательная реализация;
			\item Параллельная реализация при помощи POSIX threads;
			\item Параллельная реализация при помощи технологии MPI.
		\end{enumerate}
		Сравнить времена выполнения программ, в зависимости от окружения, а так же от количества потоков (процессов) для вариантов 2 и 3.
	\section{Реализация}
		\subsection{Последовательное выполнение}
			\subsubsection{Описание алгоритма}
				В качестве параметров программе передается имя файла с текстом. После чего программа выдает результат в следующем формате: в первой строке вывода будет записано время выполнения программы в миллисекундах, после чего выводится количество повторений слов в формате \textit{"Слово Количество\_повторений"}.
				
				Программа работает по следующему алгоритму:
				\begin{enumerate}
					\item Анализ входных параметров для получения имени файла.
					\item Инициализируем глобальный вектор для хранения слов и map для хранения повторений слов.
					\item Считываем входной файл в массив типа char.
					\item Инициализируем таймер и получаем время начала работы программы.
					\item Последовательно берем каждое новое слово в строке, до тех пор, пока не закончится строка и:
						\begin{itemize}
							\item Если слово не находится в векторе встреченных слов, то добавляем его в вектор и добавляем в map пару типа \textit{Новое\_слово 1}.
							\item Если слово находится в векторе, то увеличиваем соответствующее значение в map на 1.
						\end{itemize}
					\item Считываем время завершения работы и находим время выполнения.
					\item Выводим время выполнения, а так же содержимое map и вектора.
				\end{enumerate}
			\subsubsection{Исходный код}
			\lstinputlisting[style={CppCodeStyle}]{../serial/main.cpp}
			\subsection{Выполнение при помощи pthreads}
				\subsubsection{Описание алгоритма}
					В качестве параметров программе передается количество потоков и имя файла с текстом. После чего программа выдает результат в следующем формате: в первой строке вывода будет записано время выполнения программы в миллисекундах, после чего выводится количество повторений слов в формате \textit{"Слово Количество\_повторений"}.
					
					Программа работает по следующему алгоритму:
					\begin{enumerate}
						\item Анализ входных параметров для получения имени файла и количество потоков.
						\item Инициализируем глобальный вектор для хранения слов и map для хранения повторений слов, а так же мьютекс для обеспечения совместного доступа к глобальным переменным.
						\item Считываем входной файл в массив типа char.
						\item Инициализируем таймер и получаем время начала работы программы.
						\item Разбиваем входной массив на n (n - количество потоков), массивов примерно одинаковой длинны. Для этого входную строку делим на n равных частей, после чего смещаем каждую границу до первого разделяющего символа (пробела, точки, запятой и т.д.).
						\item Для каждого из потоков запускаем функцию подсчета количества слов и переводим его в отсоединенный режим:
						\begin{enumerate}
							\item Инициализируем локальные вектор и карту для подсчета слов.
							\item Последовательно берем каждое новое слово в строке, до тех пор, пока не закончится строка и:
							\begin{itemize}
								\item Если слово не находится в векторе встреченных слов, то добавляем его в вектор и добавляем в map пару типа \textit{Новое\_слово 1}.
								\item Если слово находится в векторе, то увеличиваем соответствующее значение в map на 1.
							\end{itemize}
							\item Переводим мьютекс в заблокированное состояние.
							\item Объединяем локальные вектор и карту с глобальными вектором и картой.
							\item Разблокируем мьютекс.
						\end{enumerate}
						\item Ожидаем завершения всех потоков.
						\item Считываем время завершения работы и находим время выполнения.
						\item Выводим время выполнения, а так же содержимое map и вектора.
					\end{enumerate}
				\subsubsection{Исходный код}
					\lstinputlisting[style={CppCodeStyle}]{../parallel/main.cpp}
			\subsection{Выполнение при помощи mpi}
				\subsubsection{Описание алгоритма}
					В качестве параметров программе передается количество процессов и имя файла с текстом. После чего программа выдает результат в следующем формате: в первой строке вывода будет записано время выполнения программы в миллисекундах, после чего выводится количество повторений слов в формате \textit{"Слово Количество\_повторений"}.
					
					Программа работает по следующему алгоритму:
					\begin{enumerate}
						\item Анализ входных параметров для получения имени файла и количества процессов.
						\item Инициализируем глобальный вектор для хранения слов и map для хранения повторений слов.
						\item Считываем входной файл в массив типа char.
						\item Инициализируем таймер и получаем время начала работы программы.
						\item При помощи функции \textit{MPI\_Init} разбиваем процесс на несколько процессов. При этом процесс с ID 0 будет "мастером", а остальные процессы будут "служебными". После чего для каждого из типов процессов будет свой алгоритм выполнения.
						\item Процесс-мастер:
							\begin{enumerate}
								\item Разбиваем входную строку на n (n - количество служебных процессов) по такому же принципу, как в многопоточной программе.
								\item Передаем каждому из служебных процессов последовательно два сообщения:
									\begin{itemize}
										\item Размер строки, которую собираемся передать.
										\item Подстроку, которую будет обрабатывать этот служебный процесс.
									\end{itemize}
								\item Получаем от каждого из служебного процесса последовательно следующие сообщения:
									\begin{itemize}
										\item Размер строки, которую собираемся передать.
										\item Подстроку с результатом подсчета слов формата \textit{Слово Количество\_повторений}.
									\end{itemize}
								\item Разбираем строку и обновляем глобальные вектор и карту по аналогии с последовательной программой.
								\item Выводим время завершения обработки.
								\item Выводим результаты.
							\end{enumerate}
						\item Служебный процесс:
							\begin{enumerate}
								\item Получаем от процесса мастера сообщение с длинной строки, которую необходимо принять и инициализируем память по эту строку.
								\item Получаем строку с текстом.
								\item Инициализируем локальные вектор и карту для подсчета слов.
								\item Последовательно берем каждое новое слово в строке, до тех пор, пока не закончится строка и:
									\begin{itemize}
										\item Если слово не находится в векторе встреченных слов, то добавляем его в вектор и добавляем в map пару типа \textit{Новое\_слово 1}.
										\item Если слово находится в векторе, то увеличиваем соответствующее значение в map на 1.
									\end{itemize}
								\item Превращаем вектор и карту в строку вида \textit{"Слово Количество\_повторений"}.
								\item Отправляем процессу-мастеру сообщение с длинной полученной строки.
								\item Отправляем процессу-мастеру сообщение с созданной строкой.
							\end{enumerate}
					\end{enumerate}
				\subsubsection{Исходный код}
					\lstinputlisting[style={CppCodeStyle}]{../mpi/main.cpp}
		\section{Тестирование производительности программ}
			Для автоматизации тестирования производительности было решено написать следующий скрипт:
			\begin{verbatim}
				#!/bin/bash
				
				RDIR=`pwd`
				TEST_DIR="$RDIR/testFiles"
				TEST_FILES="test1.txt"
				REPORT_DIR="$RDIR/testReports"
				
				# Количество повторений
				let COUNTER_VAR=50
				
				# Подотовка директорий
				
				echo "Create report dir $RDIR"
				
				if [ -d $REPORT_DIR ] ; then
				rm -R $REPORT_DIR
				fi
				mkdir $REPORT_DIR
				
				# Сборка задач
				make clean
				make
				
				# Запуск задач
				for i in $TEST_FILES ; do
				echo "Start serial programm for test file $i"
				$RDIR/workSerial $TEST_DIR/$i > $REPORT_DIR/"$i".result.serial
				echo "Start thread programm for test file $i"
				$RDIR/workThreads 4 $TEST_DIR/$i > $REPORT_DIR/"$i".result.thread
				echo "Start mpi programm for test file $i"
				mpirun -np 4 $RDIR/workMPI $TEST_DIR/$i > $REPORT_DIR/"$i".result.mpi
				done
				
				# Сравнение результатов
				for i in $TEST_FILES ; do
				cat $REPORT_DIR/"$i".result.serial | sort > $REPORT_DIR/"$i".result.serial.tmp
				mv $REPORT_DIR/"$i".result.serial.tmp $REPORT_DIR/"$i".result.serial

				cat $REPORT_DIR/"$i".result.thread | sort > $REPORT_DIR/"$i".result.thread.tmp
				mv $REPORT_DIR/"$i".result.thread.tmp $REPORT_DIR/"$i".result.thread
				
				cat $REPORT_DIR/"$i".result.mpi | sort > $REPORT_DIR/"$i".result.mpi.tmp
				mv $REPORT_DIR/"$i".result.mpi.tmp $REPORT_DIR/"$i".result.mpi
				
				diff $REPORT_DIR/"$i".result.serial $REPORT_DIR/"$i".result.thread >
				  $REPORT_DIR/"$i".diff.serial.thread
				diff $REPORT_DIR/"$i".result.serial $REPORT_DIR/"$i".result.mpi >
				  $REPORT_DIR/"$i".diff.serial.mpi
				done
				
				# Многократный запуск для последующего рассчета СКО и т.д.
				
				TEST_FILE=test1.txt
				
				# Подготовка
				find -name *.repeate | xargs rm -f
				
				# Последовательная программа
				COUNTER=0
				
				echo "Start serial programm repeating..."
				while [ $COUNTER -lt $COUNTER_VAR ] ; do
				$RDIR/workSerial $TEST_DIR/$TEST_FILE | head -n 1 >>
				  $REPORT_DIR/result.serial.repeate
				let COUNTER=COUNTER+1 
				done
				echo "Done"
				echo ""
				
				# Параллельная программа с 1 потоком
				COUNTER=0
				
				echo "Start pthreads programm with 1 thread repeating..."
				while [ $COUNTER -lt $COUNTER_VAR ] ; do
				$RDIR/workThreads 1 $TEST_DIR/$TEST_FILE | head -n 1 >>
				  $REPORT_DIR/result.threads.1.repeate
				let COUNTER=COUNTER+1 
				done
				echo "Done"
				echo ""
				
				# Параллельная программа с 2 потоками
				COUNTER=0
				
				echo "Start pthreads programm with 2 thread repeating..."
				while [ $COUNTER -lt $COUNTER_VAR ] ; do
				$RDIR/workThreads 2 $TEST_DIR/$TEST_FILE | head -n 1 >>
				  $REPORT_DIR/result.threads.2.repeate
				let COUNTER=COUNTER+1 
				done
				echo "Done"
				echo ""
				
				# Параллельная программа с 4 потоками
				COUNTER=0
				
				echo "Start pthreads programm with 2 thread repeating..."
				while [ $COUNTER -lt $COUNTER_VAR ] ; do
				$RDIR/workThreads 4 $TEST_DIR/$TEST_FILE | head -n 1 >>
				  $REPORT_DIR/result.threads.4.repeate
				let COUNTER=COUNTER+1 
				done
				echo "Done"
				echo ""
				
				# MPI с 1 рабочим процессом
				COUNTER=0
				
				echo "Start mpi programm with 1 process repeating..."
				while [ $COUNTER -lt $COUNTER_VAR ] ; do
				mpirun -np 2 $RDIR/workMPI $TEST_DIR/$TEST_FILE | head -n 1 >>
				  $REPORT_DIR/result.mpi.1.repeate
				let COUNTER=COUNTER+1 
				done
				echo "Done"
				echo ""
				
				# MPI с 2 рабочими процессами
				COUNTER=0
				
				echo "Start mpi programm with 2 process repeating..."
				while [ $COUNTER -lt $COUNTER_VAR ] ; do
				mpirun -np 3 $RDIR/workMPI $TEST_DIR/$TEST_FILE | head -n 1 >>
				  $REPORT_DIR/result.mpi.2.repeate
				let COUNTER=COUNTER+1 
				done
				echo "Done"
				echo ""
				
				# MPI с 3 рабочими процессами
				COUNTER=0
				
				echo "Start mpi programm with 4 process repeating..."
				while [ $COUNTER -lt $COUNTER_VAR ] ; do
				mpirun -np 4 $RDIR/workMPI $TEST_DIR/$TEST_FILE | head -n 1 >>
				  $REPORT_DIR/result.mpi.4.repeate
				let COUNTER=COUNTER+1 
				done
				echo "Done"
				echo ""
			\end{verbatim}
			Скрипт работает по следующему алгоритму:
			\begin{enumerate}
				\item Очищаем прошлые результаты.
				\item Подготавливаем директории для результатов теста.
				\item Выполняем сборку задач.
				\item Для каждого из тестовых файлов выполняем:
					\begin{enumerate}
						\item Запускаем последовательную программу и записываем результат выполнения в файл \textit{Тестовая\_директория/Имя\_файла.result.serial}.
						\item Запускаем параллельную программу с 4 потоками и записываем результат выполнения в файл \textit{Тестовая\_директория/Имя\_файла.result.thread}.
						\item Для проверки корректности работы программы находим разницу между результатами выполнения между последовательной и параллельной и последовательной и mpi программами и выводим их в файлы \textit{Тестовая\_директория/ Имя\_файла. diff .serial .thread} и \textit{Тестовая\_директория/ Имя\_файла .diff .serial .mpi} соответственно.
					\end{enumerate}
			\end{enumerate}
\end{document}}